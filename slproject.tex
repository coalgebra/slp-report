\documentclass{ctexart}
\CTEXsetup[format+={\flushleft}]{section}
\usepackage{listings}
\usepackage{fontspec}
\usepackage{color}
\newcommand{\plogo}{\fbox{$\mathcal{PL}$}} % Generic dummy publisher logo

\usepackage[utf8]{inputenc} % Required for inputting international characters
\usepackage[T1]{fontenc} % Output font encoding for international characters
\usepackage{fouriernc} % Use the New Century Schoolbook font

\begin{document}
\begin{titlepage} % Suppresses headers and footers on the title page

	\centering % Centre everything on the title page
	
	\scshape % Use small caps for all text on the title page
	
	\vspace*{\baselineskip} % White space at the top of the page
	
	%------------------------------------------------
	%	Title
	%------------------------------------------------
	
	\rule{\textwidth}{1.6pt}\vspace*{-\baselineskip}\vspace*{2pt} % Thick horizontal rule
	\rule{\textwidth}{0.4pt} % Thin horizontal rule
	
	\vspace{0.75\baselineskip} % Whitespace above the title
	
	{\LARGE 数字逻辑设计实验报告} % Title
	 
	\vspace{0.75\baselineskip} % Whitespace below the title
	
	\rule{\textwidth}{0.4pt}\vspace*{-\baselineskip}\vspace{3.2pt} % Thin horizontal rule
	\rule{\textwidth}{1.6pt} % Thick horizontal rule
	
	\vspace{2\baselineskip} % Whitespace after the title block
	
	%------------------------------------------------
	%	Subtitle
	%------------------------------------------------
	
     Jump Or Die 的实现细节% Subtitle or further description
	
	\vspace*{3\baselineskip} % Whitespace under the subtitle
	
	%------------------------------------------------
	%	Editor(s)
	%------------------------------------------------
	
	Written By
	
	\vspace{0.5\baselineskip} % Whitespace before the editors
	
	{\scshape\Large 魏耀东 \\ 这谁啊 \\} % Editor list
	
	\vspace{0.5\baselineskip} % Whitespace below the editor list
	
	\textit{Zhejiang University} % Editor affiliation
	
	\vfill % Whitespace between editor names and publisher logo
	
	%------------------------------------------------
	%	Publisher
	%------------------------------------------------
	
	
	\vspace{0.3\baselineskip} % Whitespace under the publisher logo
	
	2017 % Publication year
	
	{\large publisher} % Publisher
\end{titlepage}
\newpage
\tableofcontents
\newpage
    \section{背景介绍}
	作为数字逻辑设计这门课程的课程设计,我们选择了制作一款基于键盘与vga显示器的
	小游戏。玩家可以通过尽量延长角色在场景中的存活时间以及拾取随机生成的奖励物品
	来获得分数。 \\
	这里应该有图
	\section{设计说明}
		\subsection{设计开发环境}
		\begin{itemize}
			\item 实验平台 : Sword 开发板
			\item 开发环境 : Xilinx ISE
			\item 硬件描述语言 : Verilog HDL
		\end{itemize}
		\subsection{交互媒介}
		\begin{itemize}
			\item 输入 : Sword 板上的小键盘、开关,以及外接的 PS2 协议键盘。
			\item 输出 : 通过 VGA 显示器显示出的图像
		\end{itemize}
        \subsection{数码管显示模块设计}
        \subsection{VGA 显示模块设计}
		\subsection{小键盘输入模块设计}
		\subsection{PS2 键盘输入模块设计}
		
    \section{模块结构}
        \subsection{图}
        \subsection{各种模块}
        \subsection{各种模块}
    \section{程序流程} 
        \subsection{程序跑}
    \section{调试过程分析}
    \section{程序功能测试}
    \section{程序代码分析}
    \section{组内分工}
        \subsection{贡献比例}
    \section{参考资料}
    
\end{document}